\section{System Architecture}
\label{sec:system_architecture}

\subsection{Overall System Design}

Our energy-efficient deep learning inference framework for legacy GPUs follows a modular, hierarchical architecture designed to maximize performance while maintaining energy efficiency. The system is organized into four primary layers, as illustrated in Figure \ref{fig:system_architecture}.

\begin{figure}[H]
\centering
\includegraphics[width=\textwidth]{figures/system_architecture.pdf}
\caption{Overall System Architecture}
\label{fig:system_architecture}
\end{figure}

\subsection{Application Layer}

The application layer serves as the primary interface for deep learning workloads. It includes:

\begin{enumerate}
    \item \textbf{Model Loader:} Supports popular deep learning frameworks (TensorFlow, PyTorch)
    \item \textbf{Inference Engine:} Optimized execution of neural network models
    \item \textbf{Workload Analyzer:} Characterizes computational patterns and resource requirements
    \item \textbf{API Interface:} Provides high-level programming interfaces for application integration
\end{enumerate}

\subsection{Optimization Layer}

The optimization layer implements our multi-algorithm approach to matrix multiplication:

\subsubsection{Algorithm Implementations}

\begin{enumerate}
    \item \textbf{Low-Rank Matrix Approximator (LRMA):}
    \begin{itemize}
        \item SVD-based dimensionality reduction
        \item Adaptive rank selection based on error tolerance
        \item Memory-efficient implementation for large matrices
    \end{itemize}

    \item \textbf{Coppersmith-Winograd Algorithm (CW):}
    \begin{itemize}
        \item Block-based matrix multiplication
        \item Reduced arithmetic complexity: $O(n^{2.376})$
        \item Cache-aware memory access patterns
    \end{itemize}

    \item \textbf{Quantum Annealing Simulator (QAS):}
    \begin{itemize}
        \item Simulated quantum optimization for matrix operations
        \item Parallel processing of subproblems
        \item Adaptive cooling schedules
    \end{itemize}

    \item \textbf{Tensor Core Emulator (TCE):}
    \begin{itemize}
        \item Software emulation of tensor operations
        \item Tile-based computation patterns
        \item Memory layout optimization
    \end{itemize}
\end{enumerate}

\subsubsection{Intelligent Technique Selector}

The technique selector employs machine learning algorithms to make optimal algorithm choices:

\begin{enumerate}
    \item \textbf{Feature Extraction:} Analyzes matrix characteristics (sparsity, condition number, dimensions)
    \item \textbf{Performance Prediction:} Uses regression models to estimate execution time and energy consumption
    \item \textbf{Bayesian Optimization:} Explores algorithm parameter spaces for optimal configurations
    \item \textbf{Adaptive Learning:} Updates selection models based on execution feedback
\end{enumerate}

\subsection{Hardware Abstraction Layer}

The hardware abstraction layer provides unified access to GPU resources:

\begin{enumerate}
    \item \textbf{Memory Management:} Efficient GPU memory allocation and transfer
    \item \textbf{Kernel Launch:} Optimized kernel execution with appropriate workgroup sizes
    \item \textbf{Synchronization:} Proper barrier management for concurrent operations
    \item \textbf{Error Handling:} Robust error detection and recovery mechanisms
\end{enumerate}

\subsection{Power Profiling Framework}

The power profiling framework provides comprehensive energy monitoring:

\subsubsection{Power Sensors}

\begin{enumerate}
    \item \textbf{GPU Power Draw:} Real-time measurement of GPU power consumption
    \item \textbf{Memory Power:} Separate tracking of memory subsystem power
    \item \textbf{System Integration:} Correlation with total system power consumption
\end{enumerate}

\subsubsection{Performance Counters}

\begin{enumerate}
    \item \textbf{Instruction Throughput:} Monitoring of arithmetic operations
    \item \textbf{Memory Bandwidth:} Tracking of memory access patterns
    \item \textbf{Cache Hit Rates:} Analysis of memory hierarchy efficiency
    \item \textbf{Branch Divergence:} Tracking of execution flow efficiency
\end{enumerate}

\subsubsection{Thermal Monitoring}

\begin{enumerate}
    \item \textbf{Junction Temperature:} GPU die temperature monitoring
    \item \textbf{Memory Temperature:} VRAM temperature tracking
    \item \textbf{Fan Speed Control:} Dynamic cooling management
\end{enumerate}

\subsection{Data Management Layer}

The data management layer handles experimental data and model updates:

\begin{enumerate}
    \item \textbf{Performance Database:} Storage of benchmarking results and performance metrics
    \item \textbf{Model Repository:} Versioned storage of trained selection models
    \item \textbf{Calibration Data:} Hardware-specific calibration parameters
    \item \textbf{Logging System:} Comprehensive logging of system events and performance data
\end{enumerate}

\subsection{System Integration}

The layers communicate through well-defined interfaces:

\begin{enumerate}
    \item \textbf{Configuration Files:} JSON-based configuration for system parameters
    \item \textbf{Shared Memory:} Efficient data sharing between components
    \item \textbf{Message Passing:} Asynchronous communication for real-time adaptation
    \item \textbf{RESTful APIs:} External interfaces for monitoring and control
\end{enumerate}

This modular architecture ensures scalability, maintainability, and extensibility while providing the performance and energy efficiency required for production deep learning inference on legacy GPUs.