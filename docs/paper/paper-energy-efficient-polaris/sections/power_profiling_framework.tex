\section{Power Profiling Framework}
\label{sec:power_profiling}

\subsection{Framework Overview}

The power profiling framework provides comprehensive energy monitoring capabilities specifically designed for AMD Polaris architecture. Unlike modern GPUs with dedicated power management units, legacy hardware requires software-based instrumentation and external monitoring to achieve accurate power profiling.

\subsection{Power Measurement Architecture}

Our framework implements a multi-level power measurement approach:

\begin{enumerate}
    \item \textbf{Hardware-Level Monitoring:} Direct access to GPU power sensors
    \item \textbf{Software-Level Instrumentation:} Application-level power tracking
    \item \textbf{System-Level Correlation:} Integration with platform power consumption
\end{enumerate}

\subsection{GPU Power Sensors}

\subsubsection{AMDGPU Driver Interface}

We utilize the AMDGPU driver interfaces to access hardware power sensors:

\begin{lstlisting}[language=C, caption=GPU Power Sensor Access]
#include <amdgpu.h>
#include <amdgpu_drm.h>

// Initialize GPU context
struct amdgpu_device *adev;
amdgpu_device_initialize(fd, &adev);

// Read power consumption
struct amdgpu_power_info power_info;
amdgpu_get_power_info(adev, &power_info);

// Extract power metrics
float gpu_power = power_info.current_gpu_power;
float memory_power = power_info.current_memory_power;
\end{lstlisting}

\subsubsection{Sensor Calibration}

Due to variations in hardware and driver implementations, we implement sensor calibration:

\begin{enumerate}
    \item \textbf{Baseline Measurement:} Establish idle power consumption
    \item \textbf{Load Characterization:} Measure power under different computational loads
    \item \textbf{Temperature Correction:} Account for thermal effects on power readings
    \item \textbf{Cross-Validation:} Compare with external power meters
\end{enumerate}

\subsection{Real-Time Power Monitoring}

\subsubsection{Sampling Strategy}

Our framework implements adaptive sampling to balance accuracy and overhead:

\begin{enumerate}
    \item \textbf{High-Frequency Sampling:} 1000 Hz during kernel execution
    \item \textbf{Adaptive Resolution:} Dynamic adjustment based on power variability
    \item \textbf{Event-Driven Sampling:} Triggered by computational phase changes
\end{enumerate}

\subsubsection{Power Trace Collection}

Power traces are collected with temporal synchronization:

\begin{lstlisting}[language=Python, caption=Power Trace Collection]
import time
import threading

class PowerMonitor:
    def __init__(self, sampling_rate=1000):
        self.sampling_rate = sampling_rate
        self.power_trace = []
        self.timestamps = []
        self.monitoring = False

    def start_monitoring(self):
        self.monitoring = True
        self.monitor_thread = threading.Thread(target=self._monitor_loop)
        self.monitor_thread.start()

    def _monitor_loop(self):
        interval = 1.0 / self.sampling_rate
        while self.monitoring:
            timestamp = time.time()
            power = self._read_gpu_power()
            self.power_trace.append(power)
            self.timestamps.append(timestamp)
            time.sleep(interval)

    def stop_monitoring(self):
        self.monitoring = False
        self.monitor_thread.join()
\end{lstlisting}

\subsection{Energy Consumption Analysis}

\subsubsection{Power Integration}

Energy consumption is calculated through numerical integration of power traces:

\begin{equation}
E = \int_{t_0}^{t_f} P(t) \, dt
\label{eq:energy_integration}
\end{equation}

Where:
\begin{itemize}
    \item $E$ is total energy consumption in joules
    \item $P(t)$ is instantaneous power consumption in watts
    \item $t_0$ and $t_f$ are start and end times
\end{itemize}

\subsubsection{Discrete Integration Methods}

For practical implementation, we employ trapezoidal integration:

\begin{equation}
E \approx \sum_{i=1}^{n} \frac{(P_i + P_{i-1})}{2} \cdot (t_i - t_{i-1})
\label{eq:trapezoidal_integration}
\end{equation}

\subsection{Performance-Energy Correlation}

\subsubsection{Metrics Definition}

We define key performance-energy metrics:

\begin{enumerate}
    \item \textbf{Energy Efficiency:} GFLOPS per watt
    \begin{equation}
    \eta = \frac{\text{GFLOPS}}{\text{Power (W)}}
    \label{eq:energy_efficiency}
    \end{equation}

    \item \textbf{Energy Delay Product:} Energy consumption normalized by performance
    \begin{equation}
    EDP = \frac{E \cdot T}{P}
    \label{eq:edp}
    \end{equation}

    \item \textbf{Power Utilization:} Ratio of peak to average power consumption
    \begin{equation}
    PU = \frac{P_{\text{peak}}}{P_{\text{avg}}}
    \label{eq:power_utilization}
    \end{equation}
\end{enumerate}

\subsection{Thermal-Power Interaction}

\subsubsection{Temperature Effects}

GPU temperature significantly affects power consumption due to:

\begin{enumerate}
    \item \textbf{Leakage Current:} Exponential increase with temperature
    \item \textbf{Fan Power:} Additional power for cooling
    \item \textbf{Thermal Throttling:} Frequency reduction to prevent overheating
\end{enumerate}

\subsubsection{Thermal Modeling}

We model the relationship between temperature and power:

\begin{equation}
P(T) = P_0 + P_{\text{dynamic}} + P_{\text{leakage}}(T)
\label{eq:thermal_power_model}
\end{equation}

Where $P_{\text{leakage}}(T)$ follows an exponential relationship with temperature.

\subsection{Power-Aware Optimization}

\subsubsection{Dynamic Voltage Scaling}

The framework supports dynamic adaptation based on power constraints:

\begin{enumerate}
    \item \textbf{Power Budget Enforcement:} Maintain operation within specified power limits
    \item \textbf{Performance Scaling:} Adjust computational intensity based on available power
    \item \textbf{Quality Adaptation:} Trade accuracy for energy efficiency when constrained
\end{enumerate}

\subsubsection{Predictive Power Management}

Using historical data, the system predicts power consumption for different algorithms:

\begin{enumerate}
    \item \textbf{Algorithm Power Models:} Regression models for power prediction
    \item \textbf{Workload Classification:} Categorization based on computational patterns
    \item \textbf{Adaptive Selection:} Choose algorithms that meet power and performance constraints
\end{enumerate}

\subsection{Framework Validation}

\subsubsection{Cross-Platform Verification}

We validate our power measurements against external instrumentation:

\begin{enumerate}
    \item \textbf{External Power Meters:} Comparison with high-precision power analyzers
    \item \textbf{Thermal Imaging:} Correlation with infrared temperature measurements
    \item \textbf{Electrical Characterization:} Validation against electrical specifications
\end{enumerate}

\subsubsection{Accuracy Assessment}

Measurement accuracy is evaluated through:

\begin{enumerate}
    \item \textbf{Calibration Error:} Deviation from reference measurements
    \item \textbf{Temporal Resolution:} Ability to capture rapid power changes
    \item \textbf{Measurement Overhead:} Impact on system performance
\end{enumerate}

This comprehensive power profiling framework enables precise energy characterization of legacy GPUs, providing the foundation for energy-efficient deep learning inference optimization.