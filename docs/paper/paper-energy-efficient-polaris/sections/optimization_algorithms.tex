\section{Optimization Algorithms}
\label{sec:optimization_algorithms}

\subsection{Algorithm Overview}

Our framework implements four distinct matrix multiplication algorithms, each optimized for different computational patterns and hardware characteristics. The selection of algorithms is motivated by the diverse nature of matrix operations in deep learning workloads.

\subsection{Low-Rank Matrix Approximation (LRMA)}

\subsubsection{Algorithm Description}

Low-rank approximation exploits the inherent low-rank structure present in many matrices, particularly those derived from neural network weight matrices and activation patterns.

\begin{equation}
\mathbf{C} \approx \mathbf{U} \mathbf{\Sigma} \mathbf{V}^T
\label{eq:svd_approximation}
\end{equation}

Where $\mathbf{U}$, $\mathbf{\Sigma}$, and $\mathbf{V}^T$ are obtained through Singular Value Decomposition (SVD).

\subsubsection{Adaptive Rank Selection}

The algorithm dynamically selects the optimal rank based on reconstruction error tolerance:

\begin{algorithm}[H]
\caption{Adaptive Rank Selection for Low-Rank Approximation}
\begin{algorithmic}[1]
\REQUIRE Matrix $\mathbf{A}$, error tolerance $\epsilon$
\ENSURE Low-rank approximation $\mathbf{A}_k$
\STATE Compute SVD: $\mathbf{A} = \mathbf{U} \mathbf{\Sigma} \mathbf{V}^T$
\STATE Initialize $k = 1$
\STATE Compute cumulative energy: $E_k = \sum_{i=1}^k \sigma_i^2 / \sum_{i=1}^n \sigma_i^2$
\WHILE{$E_k < (1 - \epsilon)$ and $k < n$}
\STATE $k = k + 1$
\STATE $E_k = \sum_{i=1}^k \sigma_i^2 / \sum_{i=1}^n \sigma_i^2$
\ENDWHILE
\STATE $\mathbf{A}_k = \mathbf{U}_k \mathbf{\Sigma}_k \mathbf{V}_k^T$
\RETURN $\mathbf{A}_k$
\end{algorithmic}
\end{algorithm}

\subsubsection{Hardware Optimization}

For GPU implementation, we optimize memory access patterns:

\begin{enumerate}
    \item \textbf{Tiled SVD:} Block-wise decomposition for large matrices
    \item \textbf{Memory Layout:} Column-major storage for efficient access
    \item \textbf{Parallel Reduction:} Concurrent singular value computation
\end{enumerate}

\subsection{Coppersmith-Winograd Algorithm (CW)}

\subsubsection{Theoretical Foundation}

The Coppersmith-Winograd algorithm achieves the theoretical lower bound for matrix multiplication complexity:

\begin{equation}
\omega < 2.376
\label{eq:cw_complexity}
\end{equation}

Where $\omega$ represents the exponent in the complexity $O(n^\omega)$.

\subsubsection{Practical Implementation}

Our implementation uses a block-based approach suitable for GPU architectures:

\begin{enumerate}
    \item \textbf{Matrix Decomposition:} Divide matrices into manageable blocks
    \item \textbf{Recursive Multiplication:} Apply CW algorithm to submatrices
    \item \textbf{Memory Management:} Optimize data movement between global and shared memory
\end{enumerate}

\subsubsection{GPU Kernel Optimization}

\begin{lstlisting}[language=CUDA, caption=CW Block Multiplication Kernel]
// Coppersmith-Winograd block multiplication
__global__ void cw_block_multiply(float* A, float* B, float* C,
                                  int block_size, int n) {
    int bx = blockIdx.x, by = blockIdx.y;
    int tx = threadIdx.x, ty = threadIdx.y;

    // Shared memory for blocks
    __shared__ float As[BLOCK_SIZE][BLOCK_SIZE];
    __shared__ float Bs[BLOCK_SIZE][BLOCK_SIZE];

    // Load blocks into shared memory
    As[ty][tx] = A[(by * block_size + ty) * n + bx * block_size + tx];
    Bs[ty][tx] = B[(bx * block_size + ty) * n + by * block_size + tx];

    __syncthreads();

    // Coppersmith-Winograd computation
    float sum = 0.0f;
    for(int k = 0; k < block_size; k++) {
        // CW-specific computation pattern
        sum += cw_multiply(As[ty][k], Bs[k][tx]);
    }

    C[(by * block_size + ty) * n + bx * block_size + tx] = sum;
}
\end{lstlisting}

\subsection{Quantum Annealing Simulator (QAS)}

\subsubsection{Algorithm Motivation}

Quantum annealing provides a novel approach to optimization problems, including matrix operations. Our simulator implements quantum-inspired optimization for matrix multiplication.

\subsubsection{Simulated Annealing Implementation}

\begin{algorithm}[H]
\caption{Quantum-Inspired Matrix Multiplication}
\begin{algorithmic}[1]
\REQUIRE Matrices $\mathbf{A}$, $\mathbf{B}$, temperature $T$
\ENSURE Result matrix $\mathbf{C}$
\STATE Initialize solution space with random matrix elements
\STATE Set initial temperature $T = T_0$
\WHILE{$T > T_{\text{min}}$ and not converged}
\FOR{each matrix element $c_{ij}$}
\STATE Generate neighbor solution by perturbing $c_{ij}$
\STATE Compute energy: $E = \| \mathbf{A}\mathbf{B} - \mathbf{C} \|_F^2$
\STATE Accept with probability: $P = e^{-\Delta E / T}$
\ENDFOR
\STATE $T = T \cdot \alpha$ \COMMENT{Cooling schedule}
\ENDWHILE
\RETURN Optimized $\mathbf{C}$
\end{algorithmic}
\end{algorithm}

\subsubsection{Parallel Optimization}

The quantum annealing simulator leverages GPU parallelism:

\begin{enumerate}
    \item \textbf{Multiple Walkers:} Concurrent optimization trajectories
    \item \textbf{Shared Memory:} Fast communication between threads
    \item \textbf{Adaptive Cooling:} Dynamic temperature schedules
\end{enumerate}

\subsection{Tensor Core Emulator (TCE)}

\subsubsection{Emulation Strategy}

Since Polaris architecture lacks dedicated tensor cores, we implement software emulation of tensor operations using existing GPU resources.

\subsubsection{Tiled Matrix Multiplication}

\begin{lstlisting}[language=CUDA, caption=Tensor Core Emulation]
// Tensor core-style tiled multiplication
__global__ void tensor_core_multiply(float* A, float* B, float* C,
                                     int M, int N, int K) {
    const int TILE_SIZE = 16;

    // Thread block coordinates
    int bx = blockIdx.x, by = blockIdx.y;
    int tx = threadIdx.x, ty = threadIdx.y;

    // Shared memory tiles
    __shared__ float tileA[TILE_SIZE][TILE_SIZE];
    __shared__ float tileB[TILE_SIZE][TILE_SIZE];

    float sum = 0.0f;

    // Loop over tiles
    for(int t = 0; t < (K + TILE_SIZE - 1) / TILE_SIZE; t++) {
        // Load tiles
        if(by * TILE_SIZE + ty < M && t * TILE_SIZE + tx < K)
            tileA[ty][tx] = A[(by * TILE_SIZE + ty) * K + t * TILE_SIZE + tx];
        else
            tileA[ty][tx] = 0.0f;

        if(t * TILE_SIZE + ty < K && bx * TILE_SIZE + tx < N)
            tileB[ty][tx] = B[(t * TILE_SIZE + ty) * N + bx * TILE_SIZE + tx];
        else
            tileB[ty][tx] = 0.0f;

        __syncthreads();

        // Compute tile product
        for(int k = 0; k < TILE_SIZE; k++) {
            sum += tileA[ty][k] * tileB[k][tx];
        }

        __syncthreads();
    }

    // Store result
    if(by * TILE_SIZE + ty < M && bx * TILE_SIZE + tx < N)
        C[(by * TILE_SIZE + ty) * N + bx * TILE_SIZE + tx] = sum;
}
\end{lstlisting}

\subsubsection{Memory Layout Optimization}

The tensor core emulator optimizes memory access patterns:

\begin{enumerate}
    \item \textbf{Swizzled Layout:} Improved cache locality
    \item \textbf{Bank Conflict Avoidance:} Optimized shared memory access
    \item \textbf{Coalesced Access:} Aligned global memory transactions
\end{enumerate}

\subsection{Algorithm Selection Framework}

\subsubsection{Feature Extraction}

The system extracts relevant features for algorithm selection:

\begin{enumerate}
    \item \textbf{Matrix Properties:} Dimensions, sparsity, condition number
    \item \textbf{Hardware State:} Memory availability, temperature, power budget
    \item \textbf{Performance History:} Previous execution results
    \item \textbf{Accuracy Requirements:} Acceptable error tolerance
\end{enumerate}

\subsubsection{Machine Learning Model}

We employ a multi-class classification approach for algorithm selection:

\begin{enumerate}
    \item \textbf{Training Data:} Comprehensive benchmarking results
    \item \textbf{Features:} Matrix characteristics and hardware state
    \item \textbf{Labels:} Optimal algorithm for each scenario
    \item \textbf{Model:} Random Forest classifier with feature importance analysis
\end{enumerate}

\subsubsection{Adaptive Learning}

The selection model continuously improves through feedback:

\begin{enumerate}
    \item \textbf{Performance Monitoring:} Track actual vs. predicted performance
    \item \textbf{Model Updates:} Incremental learning from new data
    \item \textbf{Confidence Estimation:} Uncertainty quantification for recommendations
\end{enumerate}

This comprehensive algorithm suite, combined with intelligent selection, enables optimal matrix multiplication performance across diverse computational scenarios on legacy AMD Polaris hardware.