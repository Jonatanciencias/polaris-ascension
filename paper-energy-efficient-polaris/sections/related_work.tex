\section{Related Work}
\label{sec:related_work}

\subsection{Energy-Efficient GPU Computing}

The field of energy-efficient GPU computing has evolved significantly in recent years. Zhang et al. \cite{zhang_energy_2014} proposed DVFS-based power management for GPUs, demonstrating up to 20\% energy savings through dynamic voltage and frequency scaling. More recently, Chen et al. \cite{chen_power_2020} developed power-aware scheduling algorithms for heterogeneous computing systems, achieving optimal performance per watt ratios.

\subsection{Matrix Multiplication Optimization}

Matrix multiplication represents a cornerstone of deep learning computations. The Coppersmith-Winograd algorithm \cite{coppersmith_winograd_1990} provides theoretical improvements over traditional approaches, though practical implementations remain challenging. Low-rank approximation techniques have been extensively studied for dimensionality reduction in neural networks \cite{sainath_low_rank_2013}.

Recent work by Dongarra et al. \cite{dongarra_algorithms_2018} provides comprehensive analysis of high-performance matrix multiplication algorithms across different architectures. Their work demonstrates that algorithm selection significantly impacts performance, particularly on GPUs with different memory hierarchies.

\subsection{Legacy Hardware Optimization}

The optimization of legacy hardware for modern workloads has gained attention as sustainability becomes a key concern. Wang et al. \cite{wang_legacy_2019} demonstrated that legacy GPUs can achieve competitive performance through software optimization, extending hardware lifespan and reducing electronic waste.

\subsection{Power Profiling Frameworks}

Power profiling for GPUs has evolved from basic monitoring to sophisticated frameworks. The NVIDIA Management Library (NVML) \cite{nvidia_nvml} provides comprehensive power monitoring for modern NVIDIA GPUs. However, equivalent tools for AMD hardware, particularly legacy architectures, remain limited.

Recent work by Luo et al. \cite{luo_power_2021} developed cross-platform power monitoring frameworks, though their focus on modern hardware limits applicability to legacy systems. Our work extends these efforts by providing detailed power profiling specifically for AMD Polaris architecture.

\subsection{Machine Learning for Algorithm Selection}

The application of machine learning for algorithm selection has shown promising results. Wang et al. \cite{wang_ml_selection_2022} used reinforcement learning to select optimal algorithms for different computational patterns. Our approach builds upon this work by incorporating hardware-specific characteristics and real-time performance feedback.

\subsection{Gap Analysis}

While significant progress has been made in GPU power management and algorithm optimization, several gaps remain:

\begin{enumerate}
    \item \textbf{Legacy Hardware Focus:} Most power profiling frameworks target modern GPUs, leaving legacy hardware underserved.

    \item \textbf{AMD Architecture Coverage:} Limited research focuses specifically on AMD GPU architectures, particularly consumer-grade hardware.

    \item \textbf{Real-time Adaptation:} Existing systems often rely on offline profiling, limiting their ability to adapt to dynamic workloads.

    \item \textbf{End-to-End Integration:} Few systems provide complete integration from algorithm selection to power-aware execution.
\end{enumerate}

Our work addresses these gaps by providing a comprehensive framework specifically designed for legacy AMD GPUs, incorporating real-time monitoring and adaptive optimization.